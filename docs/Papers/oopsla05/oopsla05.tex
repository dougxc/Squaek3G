% This is sigproc-sp.tex -FILE FOR V2.6SP OF ACM_PROC_ARTICLE-SP.CLS
% OCTOBER 2002
%
% It is an example file showing how to use the 'acm_proc_article-sp.cls' V2.6SP
% LaTeX2e document class file for Conference Proceedings submissions.
% ----------------------------------------------------------------------------------------------------------------
% This .tex file (and associated .cls V2.6SP) *DOES NOT* produce:
%       1) The Permission Statement
%       2) The Conference (location) Info information
%       3) The Copyright Line with ACM data
%       4) Page numbering
%
%  However, both the CopyrightYear (default to 2002) and the ACM Copyright Data
% (default to X-XXXXX-XX-X/XX/XX) can still be over-ridden by whatever the author
% inserts into the source .tex file.
% e.g.
% \CopyrightYear{2003} will cause 2003 to appear in the copyright line.
% \crdata{0-12345-67-8/90/12} will cause 0-12345-67-8/90/12 to appear in the copyright line.
%
% ---------------------------------------------------------------------------------------------------------------
% It is an example which *does* use the .bib file (from which the .bbl file
% is produced).
% REMEMBER HOWEVER: After having produced the .bbl file,
% and prior to final submission,
% you need to 'insert'  your .bbl file into your source .tex file so as to provide
% ONE 'self-contained' source file.
%
% Questions regarding SIGS should be sent to
% Adrienne Griscti ---> griscti@acm.org
%
% Questions/suggestions regarding the guidelines, .tex and .cls files, etc. to
% Gerald Murray ---> murray@acm.org
%
% For tracking purposes - this is V2.6SP - OCTOBER 2002

%\documentclass{acm_proc_article-sp}
\documentclass{sig-alt-sun}
\usepackage{url}

\begin{document}
%
% --- Author Metadata here ---
\conferenceinfo{OOPSLA'05,}{October 16--20, 2005, San Diego, California, USA.}
\CopyrightYear{2005}
\crdata{1-59593-193-7/05/0010}
% --- End of Author Metadata ---

\title{The Squawk Virtual Machine: Java(TM) on the Bare Metal}
\subtitle{[Extended Abstract]}

\numberofauthors{2}
\author{
\alignauthor Doug Simon \\
       \affaddr{Sun Microsystems Laboratories}\\
       \affaddr{16 Network Drive}\\
       \affaddr{Menlo Park, CA 94025}\\
       \email{doug.simon@sun.com}
\alignauthor Cristina Cifuentes \\
       \affaddr{Sun Microsystems Laboratories}\\
       \affaddr{16 Network Drive}\\
       \affaddr{Menlo Park, CA 94025}\\
       \email{cristina.cifuentes@sun.com}
}
\date{}
\maketitle

\begin{abstract}
The Squawk virtual machine is a small Java(TM) VM written in 
Java that runs without an OS on small devices.
Squawk implements an isolate mechanism 
%similar to JSR-121 
allowing applications to be reified.  Multiple isolates 
can run in the one VM, and isolates can be migrated between
different instances of the VM.

%Squawk can run multiple applications at once while
%ensuring isolation between the applications. 
%This results in saving precious space on small devices. 
%The applications
%themselves can be reified (i.e. applications can be treated
%as first class objects) by means of an API similar to the JSR-121
%Isolate API.  The applications can also be migrated between
%different instances of the Squawk VM.
\end{abstract}

\category{D.3.3}{Programming Languages}{Language Constructs and Features}[classes and objects]
\category{D.3.4}{Programming Languages}{Processors}[interpreters, run-time environments]

\terms{Languages}


\section{Introduction}

Java virtual machines are typically written in C/C++. However, a number of
complex processes performed by a JVM can be better expressed in Java, which
offers features such as type safety, garbage collection and exception handling.
This technique has been adopted by a few other JVMs, as well as VMs for
other languages such as Smalltalk\cite{Inga97}. Implementing most of the
VM in Java also eases the porting and debugging of the VM.  For small devices, 
Java bytecodes typically offer a space saving over natively compiled
code.

%On a small device, it is desirable to reduce the redundancy of
%operations performed by the OS and a JVM.  One means for achieving this is to
%remove the OS completely and have the JVM provide all the required
%functionality. Such an architecture requires providing a mechanismm
%for handling interrupts and writing device drivers in Java.
%
%This extended abstract describes the architecture of the
%Squawk Virtual Machine, a JVM written in Java that provides OS level 
%functionality.

On small devices, providing OS functionality in the JVM allows for 
a simpler, more compact VM/OS that supports handling interrupts and
writing device drivers in Java.  Other OS functionality such as networking
stack and resource management need also be provided by the VM.  
This extended abstract describes our approach at providing a solution 
to this problem space.

\vfill\eject
\section{The Squawk Virtual Machine}

The Squawk JVM is the result of an effort to write a CLDC compliant
JVM in Java which also provides OS level mechanisms for small 
devices.  
Squawk came out of earlier similar efforts at Sun Labs on systems such 
as the KVM~\cite{Taiv99}.
A lot of its design was driven by the insight that performing up front
transformations on Java bytecode into a more friendly execution format
can greatly simplify other parts of the VM. 

The Squawk architecture can be thought of as a split VM with the classfile
preprocessor, called the {\em translator}, on one end and the execution
engine on the other. The translator produces a more compact
version of the input Java bytecodes that have the following properties: 
1- symbolic references to other classes, fields and methods have been
resolved;
% into (indirect) pointers, object offsets and method table offsets respectively; 
2- local variables are re-allocated such that slots 
are partitioned to hold only pointer or non-pointer values; 
%values or only non-pointer values; 
%and 3- By means of inserted spills and fills, 
and 3- the operand stack is guaranteed
to be empty for certain instructions whose execution may result in a
memory allocation. 

The last two transformations greatly simplify garbage collectors, as each 
method only requires a single pointer map and there is no need to scan the 
operand stack.  
%This contrasts with the mechanism used in other small VMs (e.g., KVM) where a pointer
%map for the local variable slots and operand stack slots is recorded
%at certain control flow points in each method.  Not only do these
%{\em stackmaps} occupy more space, the collector must still
%do a static execution of the code in a method if the current
%execution point does not exactly correspond with one of the stackmaps.

%The Squawk JVM optimizes a number of its core data structures to save space.
%Non-array objects have a single word header and array objects have
%a two word header. Methods are encoded as a special type of byte array
%with a variable length object header containing the information required
%for executing the method (i.e., exception tables, pointer to defining
%class, number of parameters and local variables, etc.) and the body of the
%array are the bytecodes themselves. Strings containing only ASCII
%characters are encoded as a special type of byte array and all other
%strings are encoded as a special type of char array. The symbolic
%information for a set of classes (i.e., the names and signatures of fields and methods)
%can be stripped to varying degrees by the translator, or discarded entirely
%if they are never to be linked against by classes in a subsequent translation.
%The symbolic information that is retained is encoded in a single string.

\section{Object Memory Serialization \\ and Suite Files}

The Squawk JVM includes a mechanism for serializing a graph of objects.
It is very similar to the mark phase of a garbage collector
and is actually implemented on top of the collector. All the pointers
in the serialized object graph are relocated to canonical addresses.
The serialized form can be
deserialized back into a live object graph at a later time.

The output of the translator is a collection of internal class
data structures encapsulated in an object called a suite. The
translator loads a transitive closure of classes. 
%(a requirement for
%the eager resolution of symbolic references), and every class
%in a suite will only refer to other classes in the same suite or
%to a previously loaded suite. That is, 
Suites can form a chain where
each subsequent suite only refers to previous suites.

By combining the chained nature of suites with the object serialization
mechanism, Squawk can save a set of loaded and translated
classes to a file. These suite files can be loaded into a subsequent
execution of Squawk, providing a faster alternative to
standard classfile loading, greatly improving VM startup time.
%Relocation is simply a single pass over the object memory using
%a pointer map that was saved with the suite.

The ability to save suites to files also enables Squawk to
operate as a split VM for deployment of classfiles to devices that
do not have the resources required for classfile loading. 
On average, suite files are 35\% the size of classfiles.


%\pagebreak
\section{The Isolation Mechanism}

Application isolation is implemented by placing application specific
state such as class initialization state and class variables inside
an {\em isolate}. The VM is always
executing in the context of a single {\em current} isolate and
access to this state is indirected to the relevant data in the
isolate object.

This indirection prevents two applications from interfering
with each other via access to class variables or even by synchronizing
on shared immutable data structures.

\subsection*{Isolate Migration}

The object memory serialization mechanism used to save suites to a file
can also be used to externalize the running state of an application.
This capability allows checkpointing of applications. It has also been
used to achieve migration of a running isolate from one VM to another
over a network connection. The serialization/deserialization process
also works when migrating between machines with different endianness.

However, true migration of an application's state is in general a very
hard, if not impossible problem, given that a substantial amount of
state may not be under complete control of the VM (e.g., open socket
connections). The Squawk JVM sidesteps these issues by simply throwing
a (catchable) exception if an application has open connections when
an attempt is made to migrate it.

\section{Interrupts}

Squawk supports interrupts written in Java. 
When an interrupt occurs, a low level 
assembler routine disables the source of an interrupt and sets a bit in an
interrupt status word (ISW). An interrupt handler thread
is blocked on an event correlated with the bit in the ISW.
At each reschedule point, the scheduler resumes the interrupt 
handler thread which handles the interrupt, and reenables it. 

% The driver is responsible for
%clearing the relevant bits in the ISW and re-enabling the interrupt source.

This design had an unpredictable latency due to the 
non-preemptible garbage collector in Squawk. To address this,
the VM was extended to support two separate contexts, each 
with its own heap and processor state registers.
Interrupts now cause an immediate switch to the single-threaded kernel context,
which sequentially executes all the relevant blocked interrupt handlers. 
{\em This mechanism is work in progress}. 

\section{Case Study: Squawk on the Sun SPOT Platform}

The Sun SPOT platform is an experimental wireless transducer 
platform under development at Sun Labs.  Transducers; sensors combined
with actuator mechanisms; are commonly used in sensors, robots, home
appliances, motors, and other devices.
The Sun SPOT platform includes
an ARM-7 with 256 Kb of RAM and 2 Mb of flash, an 802.15.4 radio,
and a general purpose sensor application board (with a 3D accelerometer, 
a temperature sensor, a light sensor, two LEDs and two switches).  The device
can be powered from a variety of power sources, including 1.5V batteries.

The interpreted version of Squawk is deployed on the Sun SPOT platform. 
%This version uses the polling-based interrupt architecture. 
The core VM is 80~Kb of RAM and the 
libraries (CLDC~1.0, 802.15.4, and hardware and sensor integration/control) 
are 270~Kb of flash.  All device drivers and the 802.15.4 MAC layer are 
written in Java. 

The performance of Squawk on the Sun SPOT platform is comparable 
to that of the KVM; an interpreted JVM written in C that runs on top of an OS.  
Further, deployed Java applications in the form of suite files are about one 
third the size of their classfile counterpart.


\section{Comparison to Other Work}
%Few Java VMs have been written in Java, even fewer can run on 
%the bare metal without need for an underlying OS.  
%Squawk is written in Java and runs on the bare metal on the ARM.  
%Its small size allows it to run on wireless transducer devices, as 
%well as cell phones. 

IBM's Jikes Research Virtual Machine(RVM), formerly known as
the Jalape\~{n}o virtual machine~\cite{Alpe99}, and the 
Ovm project~\cite{Pala03}, are Java VMs written in Java. 
%They both make use of the GNU classpath.  
They both run desktop and server applications, and require an OS to run.  
Ovm implements the real-time specification for Java (RTSJ) and
has been used by Boeing to test run an unmanned plane. 

JX~\cite{Golm02} and Jnode~\cite{Lohm05} are Java operating 
systems that implement a Java VM as well as an OS.  Security 
in the OS is provided through the typesafety of the Java bytecodes. 
JX runs on the bare metal on x86 and PPC, and Jnode on the x86.
They both access IDE disks, video cards and NICs.
JX runs on cell phones and desktops.  
Jnode runs desktop applications.

\section{Conclusions}
Squawk is a Java VM written in itself, designed for small devices, 
and servicing hardware interrupts.
Squawk is currently deployed on the Sun SPOT platform, an 
ARM-7 based wireless transducer device where all device 
drivers are written in Java.
%The core VM is 80~KB of RAM and its 
%performance is similar to that of the KVM, a JVM written in 
%C that runs on top of an OS. 

%Squawk has been designed to run on small devices. 
Squawk implements an isolation mechanism, can run multiple 
applications in the one VM, and can migrate applications 
to another machine that runs the same VM. 

\subsection*{Acknowledgments}
The original design and implementation of Squawk was
due to Nik Shaylor.
For more information on Squawk refer to 
\url{http://research.sun.com/projects/squawk}

%We would like to thank Olaf Manczak, Randy Smith, Andrew Crouch
%and David Liu for suggestions on improving the presentation
%of the poster, and Nancy Snyder for helping with the poster's 
%graphics and layout.

\bibliographystyle{abbrv}
\bibliography{vm}

\end{document}
