%\documentclass[preprint]{sigplanconf}
\documentclass{sigplanconf}
\usepackage{epsfig}
\usepackage{url}

\newcommand{\centerfigbegin}{\begin{figure}[htp] \begin{center}}
\newcommand{\centerfigend}[2]{{\caption{\label{#1} {#2}}} \end{center} \end{figure} }

\newcommand{\doublecenterfigbegin}{\begin{figure*}[htp] \begin{center}}
\newcommand{\doublecenterfigend}[2]{{\caption{\label{#1} {#2}}} \end{center} \end{figure*} }

\newcommand{\psfigbegin}[2]{\begin{figure}[htp] \centerline{\psfig{figure={#1},height={#2}}} }
\newcommand{\psfigend}[2]{{\small \bf \caption{\label{#1} {#2}}} \end{figure} }

\newcommand{\doublepsfigbegin}[2]{\begin{figure*}[htp] \centerline{\psfig{figure={#1},height={#2}}} }
\newcommand{\doublepsfigend}[2]{{\small \bf \caption{\label{#1} {#2}}} \end{figure*} }

\newcommand{\centertablebegin}{\begin{table}[htp] \begin{center}}
\newcommand{\centertableend}[2]{{\bf \caption{\label{#1} {#2}}} \end{center} \end{table} }

\newcommand{\doublecentertablebegin}{\begin{table*}[htp] \begin{center}}
\newcommand{\doublecentertableend}[2]{{\bf \caption{\label{#1} {#2}}} \end{center} \end{table*} }

\def\remark#1{\marginpar{\raggedright\hbadness=10000
        \def\baselinestretch{0.8}\scriptsize
        \it #1\par}}


\begin{document}

\conferenceinfo{VEE'06}{June 14--16, Ottawa, Ontario, Canada}
\copyrightyear{2006}
\copyrightdata{1-59593-332-6/06/0006}
\Sunpermission

%\titlebanner{DRAFT - Do not distribute}
%\preprintfooter{Paper for VEE'06}

\title{Java\texttrademark\ on the Bare Metal of Wireless Sensor Devices}
\subtitle{The Squawk Java Virtual Machine}
\authorinfo{Doug Simon}
	{Sun Microsystems Laboratories \\
        16 Network Drive \\
	Menlo Park CA 94025, USA} 
	{\url{doug.simon@sun.com}}
\authorinfo{Cristina Cifuentes}
	{Sun Microsystems Laboratories\\
	Level 10, 80 Albert Street \\ 
	Brisbane QLD 4000, Australia}
        {\url{cristina.cifuentes@sun.com}}
\authorinfo{Dave Cleal}
	{Syntropy Limited \\
	2 Stambourne Way, West Wickham \\
	Kent BR4 9NF, UK}
	{\url{dave@syntropy.co.uk}}
\authorinfo{John Daniels}
	{Syntropy Limited \\
	2 Stambourne Way, West Wickham \\
	Kent BR4 9NF, UK}
	{\url{jd@syntropy.co.uk}}
\authorinfo{Derek White}
	{Sun Microsystems Laboratories \\
	One Network Drive \\
	Burlington MA 01803, USA}
	{\url{derek.white@sun.com}}
\maketitle


\begin{abstract}
The Squawk virtual machine is a small Java\texttrademark\ virtual machine (VM) 
written mostly in Java that runs without an operating system on a wireless 
sensor platform. 
Squawk translates standard class file into an internal
pre-linked, position independent format that is compact and
allows for efficient execution of bytecodes that have been placed into
a read-only memory.
In addition, Squawk implements an application isolation mechanism 
whereby applications are represented as object and are therefore treated 
as first class objects (i.e., they can be reified).  
Application isolation also enables Squawk to run multiple applications at once
with all immutable state being shared between the applications.
Mutable state is not shared.
The combination of these features reduce the memory footprint
of the VM, making it ideal for deployment on small devices.

Squawk provides a wireless API that allows developers to write 
applications for wireless sensor networks (WSNs), this API is an 
extension of the generic connection framework (GCF).  
Authentication of deployed files on the wireless device and 
migration of applications between devices is also performed by
the VM. 

This paper describes the design and implementation of the Squawk 
VM as applied to the Sun\texttrademark\ Small Programmable Object Technology 
(SPOT) wireless device; a device developed at Sun Microsystems 
Laboratories for experimentation with wireless sensor and actuator
applications.  
\end{abstract}

\category{D.3.4}{Programming Languages}{Processors}[interpreters, run-time environments]
\category{D.3.3}{Programming Languages}{Language Constructs and Features}[classes and objects]
\category{D.4.7}{Operating Systems}{Organization and Design}[real-time systems and embedded systems]
\category{D.4.6}{Operating Systems}{Security and Protection}[authentication]
\category{D.2.5}{Software Engineering}{Testing and Debugging}[debugging aids]

\terms
Languages, Experimentation, Security

\keywords
Embedded systems, Java virtual machine, Sun SPOT, IEEE 802.15.4, Wireless sensor networks


\section{Introduction}

The pervasive computing vision depicts a future in which computation is
widely embedded in the everyday world, like ``smart dust''.  
One medium for enabling this vision is the tiny, wireless computer that 
connects to the world with sensors and actuators.
% (aka transducers).  
Despite the large interest in the area, few significant applications 
have been written so far, we believe due to the lack of adequate tools 
and languages to aid in the prototyping of applications for that domain. 

Processors associated with wireless sensor devices typically provide small 
amounts of memory, making it hard for managed runtime languages like Java 
to run on these devices, due to the static memory footprint of the virtual 
machine (VM) and the dynamic footprint of the runtime and the applications.
Traditionally, wireless sensor applications use languages such as C and assembler 
to overcome the memory limitations, at the expense of longer application
development time.  However, it is widely accepted that development time using 
managed runtime languages is less than that of non-managed languages.
Sensors and actuators are commonly used in robots, home appliances 
such as washing machines and microwaves, industrial appliances such as 
motors, set-top boxes, and many more. 

At Sun Microsystems Laboratories, we have been investigating wireless 
sensor networks by creating a next generation device we are calling the 
Sun\texttrademark\ Small Programmable Object Technology, or Sun SPOT (see 
Figure~\ref{fig-sunspot}). 
%The current Sun SPOT main board is based on an ARM7 processor and has 
%256~KB of RAM and 2~MB of flash memory, plus a separate Chipcon 2420 
%802.15.4 radio chip.  Additional sensor boards can be attached: the 
%``demo'' sensor board includes a 3-axis accelerometer, a light sensor, 
%a temperature sensor, two push buttoms, two tri-color LEDs, and 9  
%general I/O lines. 
%The device can be powered from a variety of power sources, including 
%1.5V dry cells (batteries).
%
The Sun SPOT main board is based on an ARM-9 processor and has 
512~KB of RAM and 4~MB of flash memory, plus a separate Chipcon 2420 
IEEE 802.15.4 radio chip.  Additional sensor boards can be attached: the
``demo'' sensor board includes a 3-axis accelerometer, a light sensor, 
a temperature sensor, an A/D converter, 8 tri-color LEDs, 5 general 
purpose I/O pins, and 4 hi current output pins. 

\psfigbegin{SunSPOT.eps}{4cm}
\psfigend{fig-sunspot}{Sun SPOT wireless sensor/actuator device, with a demo 
	sensor board on top, the main processor and radio board in the middle, 
	and a battery board on the bottom.}

We believe that running a managed runtime language like Java on a 
wireless sensor device will simplify application and device driver prototyping,
thereby increasing the number of developers in this domain, as well 
as their productivity; resulting in more interesting 
applications sooner.
Java brings with it garbage collection, pointer safety, exception 
handling, and a mature thread library with facilities for thread 
sleep, yield, and synchronization.  
Standard Java development and debugging tools can be used to write wireless 
sensor applications.
Further, we provide tools for deploying and monitoring these devices 
in a graphical user interface called SpotWorld~\cite{Smit05}. 
This paper concentrates on the Java virtual machine that is available
on the Sun SPOT platform: the Squawk virtual machine.

The Squawk virtual machine is a Java VM primarily written in Java and designed for
resource constrained devices.  Squawk is compliant with the 
Connected Limited Device Configuration (CLDC)~1.1 Java Micro Edition (Java ME)
configuration~\cite{CLDC11} and
runs without need for an underlying operating system; commonly referred to  
as running on the bare metal.

The initial port of Squawk to the ARM took one person 2 weeks, and a full Sun SPOT
release, including hardware, networking, and demo sensor
board libraries took two people 6 months.

By running on the bare metal, Squawk avoids the need for an
operating system (OS) in the Sun SPOT, thereby freeing up memory that would
otherwise be consumed by an OS.  
The OS functionality provided in the Squawk VM amounts to less memory 
than that required by embedded OSs such as embedded Linux. 
A lightweight configuration of embedded Linux requires 250~KB of ROM 
and 512~KB of RAM~\cite{ELinux}. 
The Squawk OS functionality includes the handling of interrupts, networking 
stack, and resource management. 
In Squawk, all device drivers and the 802.15.4 media access control (MAC) 
layer are written in Java.

A series of features made Squawk ideal for a wireless sensor platform.
The Squawk JVM:
\begin{enumerate}
\item was designed for memory constrained devices,
\item runs on the bare metal on the ARM,
\item represents applications as objects (via the isolate mechanism),
\item runs multiple applications in the one VM, 
\item migrates applications from one device to another, and 
\item authenticates deployed applications on the device.
\end{enumerate}

An earlier version of the Squawk VM was targetted at a next generation 
smart card~\cite{Shay03} which had 8~KB of RAM, 32~KB of non-volatile 
memory, and 160~KB of ROM.  
In common, both Squawk for smart card and Squawk for Sun SPOT were 
designed for memory constrained devices and therefore implement a split VM 
architecture that uses a more compact bytecode set and produces suite 
files to be loaded into the device.  
Squawk for Sun SPOT revised that design and extended it to provide support
for items 2--6 in the above list.

This paper is organized in the following way.  
$\S$\ref{sec-other-work} reviews the literature in the area. 
The design of the Squawk JVM for Sun SPOT ($\S$\ref{sec-squawk})
summarizes some of the architecture decisions made for Squawk for 
smart card, as well as expanding on the new parts of the design 
for the Sun SPOT,
implementation aspects of the suite creator and on-device VM  
are described in \S\ref{sec-translator} and \S\ref{sec-execution-engine},
$\S$\ref{sec-sunspot} describes Java programming for the
Sun SPOT.
Last, $\S$\ref{sec-results} provides some experimental results.


\section{Related Work}
\label{sec-other-work}

We review Java VMs written in Java and other VMs written in 
the language they implement.

\subsection{Java VMs Written in Java}

We review Java VMs that are written in Java as well as JVMs that
provide some OS support to run on the bare metal.

IBM's Jikes Research Virtual Machine (RVM), formerly known as
the Jalape\~{n}o virtual machine~\cite{Alpe99,Alpe00}, and the
OVM project~\cite{Pala03,Flack03}, are Java VMs written in Java.
They both make use of the GNU classpath to support desktop
and server-level Java libraries.
The GNU classpath is a series of J2SE and J2EE Java libraries
that are written in the C language.
Both Jikes and OVM run desktop and server applications, and require an OS
to run on.
OVM implements the real-time specification for Java (RTSJ) and
has been used by Boeing to test run an unmanned plane.

JX~\cite{Golm02} and Jnode~\cite{Lohm05} are Java operating
systems that implement a Java VM as well as an OS.  Security
in the OS is provided through the typesafety of the Java bytecodes.
JX runs on the bare metal on x86 and PowerPC, and Jnode on the x86.
They both access IDE disks, video cards and network interface cards.
JX runs on cell phones and desktops.
Jnode runs desktop applications.
The core of JX is written in the C language.
Neither VM makes assurances on the latency to service an interrupt,
and both allow for GC to happen while servicing interrupts.

%% This section removed as it's too over constrictive: only the programs 
%% requiring the highest level of compliance will satisfy this type of programming.
%In understanding how to provide reasonable latencies in servicing
%interrupts, work done in the real time community is of relevance.
%Typical real-time systems avoid using a GC and make use of two separate
%phases: the ``initialization phase'', where memory is allocated, and
%the ``execution phase'', where no memory is allocated and only memory
%allocated in the initialization phase is used.  In this way,
%programs do not deallocate memory. 
%\remark{Add reference?}
%
%The real-time specification for Java (RTSJ) borrows from these
%concepts and provides the concept of a memory region.  A region is
%checked at runtime to not overflow, for example, and a runtime error
%can occur.  However, a runtime error is not desirable in real-time systems, 
%as it most likely violates any real-time constraints.
%
%OVM proposes changes to regions, which are called scopes, to deal with the
%\remark{OVM reference missing}
%problem of runtime checks.  Scopes are statically checked at compile time
%rather than dynamically checked.
%Scopes impose a few new rules to make guarantees at compile time.
%There is also support for the running of existing Java code, in particular
%Java libraries, in conjuntion with scopes.
%Although it may be cumbersome to write code using these new rules, the
%argument for scopes is that only small sections of code will need to
%be written this way, as only those small sections are the ones that require
%guarantees.
%
%The Metronome project~\cite{Baco05} is researching real-time garbage
%collection for Java, aiming at pause times of less than one millisecond 
%in the worst case.
%This is an incremental GC, where garbage is collected more often
%at smaller intervals, therefore, being more suitable for real-time
%systems, where guarantees on the latency of servicing an interrupt
%are mandatory.
%%The original idea on incremental GC is due to Deutsch and Bobrow~\cite{?},
%%which dates to 1976.


\subsection{Other Language VMs Written in Their Own Language}

Smalltalk was the first object oriented language in which everything is 
built from objects.  Smalltalk was inspired by the Simula, Sketchpad, 
and Lisp languages.  
Squeak~\cite{Inga97} was the first usable implementation of Smalltalk 
written in Smalltalk itself.  The Squeak VM is written in a subset of
Smalltalk called Slang that can be translated to C.  This allows the 
VM to be written and debugged in Smalltalk, yet the translated VM performs 
well and is easy to port.  
The VM can be extended with plugins written in either Slang or C code. 

Klein~\cite{Unga05} is a VM written in the Self language that implements 
Self. 
Klein's architecture was driven by the insight that most VMs 
have three different compilation systems, making the VM complex and
hard to maintain. 
In the Klein architecture there is only need for one compiler, which
can be used statically as an ahead-of-time compiler for the system 
classes of the VM itself, and dynamically as a JIT compiler.
The Klein VM assumes memory resources typical of that on desktop machines. 


\section{The Squawk Virtual Machine}
\label{sec-squawk}

The Squawk JVM is the result of an effort to write a J2ME CLDC 
compliant JVM in Java that provides OS level mechanisms for small
devices, easing porting and debugging of the VM.
The observation was that most JVMs are written in the C and C++ 
languages, even though complex processes performed by the JVM can be 
better expressed in the Java language, which offers features such as 
type safety, garbage collection, and exception handling. 

Squawk came out of earlier similar efforts at Sun Labs on systems such
as the KVM~\cite{Taiv99}.
A large part of its design was driven by the insight that performing up front
transformations on Java bytecode into a more friendly execution format
can greatly simplify other parts of the VM.
Squawk, as its name suggests, was also inspired by the Squeak Smalltalk 
VM~\cite{Inga97}. 


\subsection{Split VM Architecture}

Resource constrained devices do not normally have enough memory
to implement class file loading on-device.  
A common design for these devices is what is known as a split 
VM architecture, namely, class file loading is performed on a 
desktop machine, the intermediate representation of the file is
then deployed onto the device, and that representation is then 
run on-device.  

\psfigbegin{SquawkSplitVM.eps}{6.5cm}
\psfigend{fig-squawk-arch}{The Squawk Split VM Architecture.  To the left is 
	the Suite Creator and to the right is the On-device VM.  
	The Suite Creator runs on the desktop and the On-device VM runs on 
	the device.  
	White boxes represent Java code, black boxes represent C code.}

Figure~\ref{fig-squawk-arch} shows Squawk's split VM architecture, 
with the class file preprocessor, called the {\em suite creator}, to the
left, and the {\em on-device VM} to the right.
The Squawk interpreter is written in C. In future, the interpreter
will be rewritten in Java and converted to C in much the same way that
the garbage collector is currently converted to C.
All other parts of the VM are written in Java. 

The suite creator transforms the Java bytecodes into a more compact 
internal representation known as the {\em Squawk bytecodes}.  These
bytecodes can be optionally optimized for code-size reduction. 
The internal object memory representation of an application can 
be serialized and saved into a file, called a {\em suite} file. 

The on-device VM interprets the suite files on-device, 
while servicing interrupts from the device itself.  

To support the split VM configuration requires a means for building and
deploying the VM's bootstrap suite onto the Sun SPOT device. The
suite creator is run in a hosted desktop VM such as HotSpot,
and all Java classes of the on-device VM, and possibly the debug
agent, are fed to the suite creator  to produce the bootstrap suite.

Note that components of the on-device VM that are either
written (interpreter) or translated (garbage collector) to C are not
processed in this form.  A separate bootloader binary (also written in
C) runs on the Sun SPOT and is responsible for receiving both these
components and the bootstrap suite and flashing them into well-known
locations on the Sun SPOT. The bootloader can then launch the
on-device VM.


\subsection{Squawk Bytecodes and Suite File Format}

The Squawk bytecodes are a compact version of Java bytecodes and 
were optimized for space, in-place execution, and to simplify garbage 
collection as follows: 

\begin{itemize}
\item Space optimization: commonly used Java bytecodes are two bytes instead 
	of three bytes.  For example, Squawk bytecodes contain 2-byte branches,
	more 1-byte load/store/const instructions, 2-byte field access, 
	and 2-byte invoke.  There is an escape mechanism in place for float 
	and double instructions, as well as widened operands.  

\item In-place execution optimization: symbolic references to other classes,
	fields, and methods are resolved into (direct) pointers, object 
	offsets, and method table offsets, respectively, eliminating the 
	constant pool and dereferencing into it. 

\item Simplification of garbage collection (GC) optimization: local variables are 
	re-allocated such that slots are partitioned to hold only pointer or 
	non-pointer values, allowing for one pointer map per method.  
	Further, the operand stack is guaranteed to contain only the operands for certain 
	instructions whose execution may result in a memory allocation; i.e.,
	empty operand stack at GC points.  This is achieved by means of 
	inserting spills and fills. 

	Both these optimizations obviate the need for stack maps and analysis during
	the collection, simplifying GC, at the expense of requiring more slots 
	in the activation records. 
	Each method only requires a single pointer map and there is no need to 
	scan the operand stack at GC points.
\end{itemize}

Suite files were designed as objects that contain a collection of 
Squawk-internal class data structures, including Squawk bytecodes.
Suite files are pre-processed sets of class file designed to be  
executed in-place (i.e., they contain position-independent bytecode). 
The serialized version of an application's object memory (a graph
of objects) is stored in the suite file; all pointers in the
serialized object graph are relocated to canonical addresses. 
Classes in a suite file can refer to classes in the same suite 
or a parent(s) suite.
For example, an application suite depends on a sensor library 
suite, which in turn depends on the VM's bootstrap suite.
This chain of suites forms a transitive class closure.  

Suite files are deserialized by the VM on-device and relocated 
by a single pass over the object memory using a pointer map that
was stored with the suite file.  This mechanism greatly improves
VM startup time and provides a faster alternative to standard 
class file loading. 

%The Squawk VM includes a mechanism for serializing a graph of objects.
%It is very similar to the mark phase of a garbage collector
%and is actually implemented on top of the collector. 

Results presented in~\cite{Shay03} when comparing the sizes of class files
as oppossed to suite files on a set of desktop benchmarks show that,
on average, suite files are 38\% the size of class files; these results 
are corroborated with the benchmarks used in this paper (see \S\ref{sec-results-filesizes}).

Note that the suite files are not compressed, this was a design 
decision made to avoid uncompressing of files on-device and to allow
in-place execution of the application without any extra overhead. 


\section{Suite Creator Implementation}
\label{sec-translator}


\subsection{Data Structures}

Squawk optimizes a number of its core data structures to save space.
In this section we describe the choices made for object layout, 
method objects, and a class' symbolic information.

\subsubsection*{Object layout}
Non-array objects have a single 32-bit word header (the class pointer) 
and array objects have a two word header (the class pointer and 
the array's length). 

All Java objects must support hashcode and monitor operations. In order 
to save space, the data associated with these relatively infrequent 
operations
is stored separately from the objects themselves. For in-RAM objects, 
this data is bundled into an {\em ObjectAssociation} object that is 
interposed between the original object and the object's class. For 
objects stored in ROM, the hashcode is a function of an object's 
address, and the monitor (if needed) is stored in a per-isolate hashtable.


\subsubsection*{Method objects}
Methods are encoded as modified byte arrays with a variable length 
header containing information needed to execute the method (e.g, 
exception tables, pointer to defining class, number of parameters 
and local variables, etc.).   The array contains the bytecode.
An encoding of 4 words is used for the common case of methods
with no handlers, less than 32 parameters, less than 32 locals, and 
less than 32 stack locations. 


\subsubsection*{String objects}
Strings containing only ASCII characters are encoded as a special type 
of byte array and all other strings are encoded with a modified type of 
char array. This is contrasted with the standard implementation of
a string as two objects (a \texttt{String} instance and a char array). As an
example, on a 32-bit platform, the string ``squawk'' occupies 16 bytes
in Squawk (8 byte header, 8 byte body) as opposed to 48 in HotSpot
(8 byte header and 16 byte body for the \texttt{String} object, and 12 byte header
and 12 byte body for the char array in the \texttt{String} object).


\subsubsection*{Symbols and metadata}
Symbols for a class (names and signatures of fields and 
methods, and access flags) are encoded in a byte array. 
The method body metadata (line number table and local variable 
tables) are stored separately. 

The symbolic information for a set of classes can be stripped 
to a varying degrees by the translator, or discarded entirely 
if they are never to be linked against by classes in a subsequent translation.


\subsection{Bytecode Optimizations}

Simple bytecode optimizations were thought useful due to the nature of the
VM code: Squawk is written using an object-oriented approach, using
setter and getter accessor methods wherever possible. Any such
accessors that are determined to be non-virtual by static analysis are
inlined. Other small static or final methods are also inlined. 

Inlining exposes more optimization opportunities, therefore, the translator
also performs simple optimizations that reduce the size of the bytecodes: 
constant folding, and constant and copy propagation. 
%dead code elimination of all kinds (dead stores, dead loads, and unreachable code).  
While all optimizations are done at a method level, the current implementation
requires that the intermediate representation for all methods be
available in order to do the inlining transformation.

The complete size-reducing benefit of these optimizations
requires dead code elimination of dead stores, dead loads, and 
unreachable code.   
This would  remove the methods that are always inlined and cannot be 
linked against by further translations (i.e. they are private or their 
symbols are stripped).
This analysis is not yet implemented in Squawk.
%Due to the meta-circular nature of the Squawk VM, there are a few VM
%methods which must never be inlined, and are annotated as such.


\subsubsection*{Verifier}

Clearly, any set of transformations can potentially introduce errors into
the translated code.  To catch some of these errors, a Squawk bytecode
verifier was designed, to verify the correctness of the generated
Squawk bytecode.
The Squawk bytecode verifier was designed to be similar to the
Java verifier. For example, it needs to check that stack sizes are
consistent across different execution paths, it needs to
determine the level of consistency of stack, local, and parameter
types across different execution paths, it needs to check that
no pops are done on an empty stack, and it needs to check type
safety.

However, the Squawk bytecode verifier has to differ from a
Java verifier due to the nature of some of the Squawk bytecodes:
some Squawk bytecodes place further constraints on the operand stack
(e.g., {\tt invokevirtual} expects only the operands of the
invocation to be on the stack),
and other bytecodes require the verifier to keep track of constant 
integer values (e.g., {\tt invokeslot} and {\tt findslot} which
replace the standard {\tt lookupswitch} Java bytecode). 
As no stack maps\footnote{
A stack map is a data structure created by the CLDC preverifier
that is stored in CLDC class files.  Stack maps specify the types
in the local variables and operand stack slots at various
points within the code. They're used to optimize verification
in a JVM by effectively making verification a one pass operation
for each method.}
are available, an iterative data flow analysis
for locals is used. 


\section{On-device VM Implementation}
\label{sec-execution-engine}

\subsection{Garbage Collection}
\label{sec-gc}

Squawk implements a mark and compact generational garbage collector, 
Lisp~2~\cite{Jone96}.
This collector is non-preemptible, which has implications for
handling interrupts in a device driver written in Java.  

%The Cheney collector is a simple and elegant collector that uses two semi-spaces 
%to copy data backwards and forwards.  It eliminates heap fragmentation, at the
%expense of doubling the address space needed in order to accommodate the second
%semi-space.  

The Lisp~2 collector uses two generations and performs three passes over the 
heap during the compaction phase.  This algorithm preserves the order of 
objects and is suitable for nodes of varying sizes. 
A sliding window is used for the young generation.  
Slices are used to obviate the need for an extra pointer-sized field in the
header of each object to store a forwarding address.
In Squawk, the same algorithm is used for marking and compaction regardless
of whether a full or partial collection is being performed.  This is a
preference for simplicity over performance. 
A bit vector is used for the entire heap: mark bits for the collection space
and write-barrier bits for the old generation.  This bit vector uses 3\% 
of the memory available for the object heap.
The collector has been extended slightly to support deep-copying of an object
graph. 

%The object memory serialization mechanism discussed in section~\ref{sec-object-memory}
%is implemented as a small addition to the configured collector
%as it is very similiar to performing a collection.

The garbage collector is written in a subset of the Java 
language and converted to C by means of a limited Java to C convertor 
based on the upcoming Java~1.6~\cite{Mustang}, which has an API for 
accessing the abstract syntax trees of the Java source compiler.
Annotations of the Java source code were used to simplify
the translation process. 

Executing the collector as compiled C code linked with the 
interpreter as opposed to Java bytecode being interpreted by 
the interpreter improved the performance of the collector by 
a factor of 10.47x on the benchmarks reported in this paper.
%
%###### Benchmark: Richards -nativegc -stats -usecgctimer ######
%1 partial collections
%1 full collections
%Full collections: [average = 29623usec, min = 29651usec, max = 29651usec]
%Partial collections: [average = 21461usec, min = 21432usec, max = 21432usec]
%Execution time was 23780 ms
%###### Benchmark: Richards -interpgc -stats -usecgctimer ######
%1 partial collections
%1 full collections
%Full collections: [average = 362839usec, min = 362838usec, max = 362838usec]
%Partial collections: [average = 166976usec, min = 166962usec, max = 166962usec]
%Execution time was 23640 ms
%###### Benchmark: DeltaBlue -nativegc -stats -usecgctimer ######
%1 full collections
%Full collections: [average = 20888usec, min = 20932usec, max = 20932usec]
%Execution time was 3151 ms
%###### Benchmark: DeltaBlue -interpgc -stats -usecgctimer ######
%1 full collections    
%Full collections: [average = 224082usec, min = 224099usec, max = 224099usec]
%Execution time was 3368 ms
%=> nativegc = 71,972usec
%   interpgc = 753,897usec
%   ratio nativegc:interpgc: 10.47x 


\subsection{Thread Scheduler}
\label{sec-threads}

Being a bare metal VM, Squawk implements green threads. 
Green threads emulate multi-threaded environments without relying on
any native operating system capabilities.  They run code in user 
space that manages and schedules threads.
Green threads switch control when control is explicitly given up by 
a thread (e.g., \texttt{Thread.yield()}), or when a thread performs a blocking
operation (e.g., \texttt{read()}). 

In Squawk, the thread scheduler supports blocking in native methods: 
a thread is blocked on an event queue polled by the scheduler, 
and events correspond to an interrupt.

Thread rescheduling is done at backward branches in application code. 
Squawk's system code is non-preemptible, which greatly simplifies
the VM design, and assumes that most time will be spent executing 
application code. 


\subsection{Interrupt Handling and Device Driver Support}
\label{sec-interrupts} 

The Squawk VM handles interrupts coming from the Sun SPOT device. 
%The device driver sets up the interrupt controller and an interrupt 
%handler thread blocks waiting for the Squawk VM to signal an 
%event.  
The device driver enables the device's interrupt.
The device driver thread blocks waiting for the VM to signal 
an event. 
When an interrupt occurs, an assembler interrupt handler sets 
a bit in an interrupt status word (ISW) and disables the 
interrupt to avoid repeats. 
%\remark{"the bit is detected" or "the ISW is polled"}
At each VM reschedule point, the ISW is checked, the event 
signalled, the scheduler resumes the device driver thread, 
which in turns handles the interrupt and (typically) re-enables 
the interrupt. 
The VM reschedules every 1,000 backward branches and when it 
wakes up, either due to an event that has happened (typically
an interrupt on the SPOT) or a timed wait that has completed 
(that is, one or more threads that were sleeping are now ready
to go).
Device drivers are written in Java making use of a peek and poke
interface to the device's memory. 

The interrupt latency is thus dependent on the time from the global interrupt 
handler running until the next VM schedule. 
This is optimal if the VM is idle.  
If the VM is executing bytecodes in another thread, the penalty is 
quite small as the VM reschedules after a certain number of backward 
branches.  There is some unpredictability in this case according to 
how close to the next reschedule the interrupt occurs.
However, if the VM is executing a garbage collection, the reschedule 
is delayed until after the GC completes, hence deteriorating the 
latency to service the interrupt.  

%In practice, the GC pause times are small enough (less than one 
%millisecond in the worst case for an 100~KB heap size) that they do 
%not cause problems for driving the devices currently
%on the Sun SPOT (e.g. radio, LED, accelerometer).
% The above data comes from Dave Cleal's test program, as per 
% described below.

In the case where a simple application has no active threads other than the
one waiting for an interrupt, we see an average latency of 0.1
milliseconds.  Where other threads are executing and creating garbage,
we've recorded worst case latencies of the order of 13 milliseconds,
although the mode is still around 0.15 milliseconds for an 100~KB heap size. 
%GC pause times when running the collector on the interpreter are up to 60
%milliseconds; an unacceptably large latency response time. 
Note that no real-time claims are made about this interrupt handling mechanism.


\section{Java Programming for the Sun SPOT}
\label{sec-sunspot}

The Sun SPOT devices combine an interesting and customizable set of
hardware features with the simplicity of Java application development. 
This section provides more details on the features of a Sun SPOT device, and 
by examples, the flavor of application development.

%The SPOT supports IEEE 802.15.4 - you can download that standard.
%ZigBee is a set of networking facilities built on top of that
%standard. The ZigBee standard is not in the public domain. We don't
%support ZigBee.


\subsection{Application Isolation}
\label{sec-isolates}

In Squawk, each application is represented by a Java object. 
This object is an instance of the class {\tt Isolate}, and can be 
used to query the status of the associated application, and
even directly affect that application through methods 
such as \texttt{start()}, \texttt{pause()}, \texttt{resume()}, and 
\texttt{exit()}, thereby allowing for the reification of applications.  

The Squawk Isolate class is an implementation of an isolation 
mechanism similar to that of Java Specification Request (JSR)~121: 
Application Isolation API Specification~\cite{JSR121,Czaj00}. 
Isolates are analogous to processes in an operating system: 
each isolate has resources that are shared amongst the 
threads of that isolate.  
In Squawk the immutable state of an isolate is shared, e.g.,
bytecode, string constants, and parts of classes. 
Non-shared class state includes static fields, class initialization
state, and class monitors. 

Isolates have a simple API: an isolate object is instantiated 
by providing the application's name and its arguments.  
For example, the following sample code instantiates an 
isolate for the \texttt{com.sun.spots.SelfHibernator} 
application and passes the url argument to the application. 
It then starts the isolate and sends an output stream to 
the isolate.

\begin{verbatim}
Isolate isolate = 
    new Isolate ("com.sun.spots.SelfHibernator", 
                 url());
isolate.start();
send (isolate, outStream);
...
\end{verbatim}

Application isolation is implemented by placing application specific
state such as class initialization state and class variables inside
the isolate object. 
The VM is always executing in the context of a single {\em current} 
isolate and access to this state is indirected to the relevant data in 
the isolate object.
This indirection prevents two applications from interfering
with each other via access to class variables or even by synchronizing
on shared immutable data structures (such as instances of
{\tt java.lang.Class}).


\subsection*{Isolate Migration}

In Squawk, applications can be checkpointed and stored to a file 
by using the serialization mechanism used with suite creation as 
well as the deep-copy support in the GC. 
During checkpointing, the status of each thread, including all 
temporary variables, can be serialized to a stream for storage.  
Because each application can be serialized to a stream, that stream
can be read into another Squawk VM to reconstitute an isolate 
on the destination device, skipping the step of storing it onto disk.  
This effectively migrates the isolate between VMs.

We have started simple experiments with isolate migration, and can 
currently move running applications from one Sun SPOT to another, 
or from one desktop or server to another. 
When moving to an architecture with a different endianness, the 
appropriate translation is performed. 

Isolate migration and checkpointing may be reminiscent of similar 
facilities available in, say, Smalltalk snapshots and the automatic
persistent memory management extension for the Spotless VM~\cite{Schn01}.  
In Smalltalk snapshots, the snapshot may wake up on a different machine, 
with different Ethernet address, different display size, and several 
other different hardware capabilities. 
In the extensions to the Spotless VM, the automatic memory management 
provided orthogonal persistence including thread state.  Java programs 
could be suspended and at a later time resumed on the same device or 
a different device, as suspended programs could be beamed between 
Palm organizers. 

In Squawk however, an individual application is the granularity of 
serialization.  Before moving, the isolate must close all open connections 
to the external world and record relevant information, so that upon 
waking it can restore the connections.  The waking isolate must sense  
the new environment, and reconnect accordingly.  In principle, it may 
not be possible to successfully connect to the new environment. 
Thus we expect isolate migration to be utilized by developers in specific
situations where such problems are known to be manageable.

%The object memory serialization mechanism used to save suites to a file
%can also be used to externalize the running state of an application.
%This capability allows checkpointing of applications. It has also been
%used to achieve migration of a running isolate from one VM to another
%over a network connection. The serialization/deserialization process
%also works when migrating between machines with different endianness.

%However, true migration of an application's state is in general a very
%hard, if not impossible problem, given that a substantial amount of
%state may not be under complete control of the VM (e.g., open socket
%connections). The Squawk JVM sidesteps these issues by simply throwing
%a (catchable) exception if an application has open connections when
%an attempt is made to migrate it.

We have added a \texttt{moveTo(IPAddress ip)} method for isolates, to facilitate 
our early experiments.  With this method, an application can itself decide 
to move from one device to another.  We expect this facility could be used 
for load balancing, or for scripting a single client server application 
that moves rather than writing two applications that connect. 
Isolate migration could be especially useful to effect an in-the-field 
replacement of one device by another (e.g., with fresh batteries) by 
letting the user simply pull the software from the old device onto the new. 
A summer intern wrote an application that migrated itself home upon 
encountering an exception, so that it could be debugged on the programmer's 
workstation before being sent back into the field. 


\subsection{Accessing Sensors in the Demo Sensor Board}

The demo sensor board library is written in Java and 
relies on peripheral classes that have been packaged in the
\texttt{com.sun.squawk.} \texttt{peripheral} domain.
The library is 400 lines of commented Java code.

An application can access the accelerometer and get its X, Y and Z 
coordinates as follows: 

\begin{verbatim}
Accelerometer3D acc =
                DemoSensorBoard.getAccelerometer();
RangeInput x = acc.getX();
RangeInput y = acc.getY();
RangeInput z = acc.getZ();
\end{verbatim}

A particular LED (numbered 1 to N) can be accessed, turned on, 
and set to a particular red, green and blue color combination in
the following way: 

\begin{verbatim}
SensorBoardColouredLED led = 
                 SensorBoardColouredLED.getLed1();
led.setOn();        
led.setRGB (50,60,10);  
\end{verbatim}

To endlessly changes the LED color by displaying either red, green or blue  
based on the direction of the accelerometer's motion, the following code
can be written:  

\begin{verbatim}
int lastX = 0,lastY = 0,lastZ = 0;
while(true) {
   int xValue = x.getValue();
   int yValue = y.getValue();
   int zValue = z.getValue();

   int r = Math.abs(xValue-lastX) > 35 ? 255:0;                    
   int g = Math.abs(yValue-lastY) > 35 ? 255:0;
   int b = Math.abs(zValue-lastZ) > 35 ? 255:0;

   led.setRGB(r,g,b);

   lastX = xValue;
   lastY = yValue;
   lastZ = zValue;
}
\end{verbatim}


\subsection{Accessing the Radio Through the Wireless API}
\label{sec-radio}

The Sun SPOT has a multilayer communications stack as shown 
in Figure~\ref{fig-radiostack}.

\psfigbegin{SunSPOTRadioStack.eps}{3cm}
\psfigend{fig-radiostack}{The Sun SPOT Radio Stack}

The two lowest levels of this stack partially implement the 802.15.4 standard. This standard is targetted at devices with low data rates (of 250 kbps, 40 kbps, and 20 kbps), with multi-month to multi-year battery life and very low complexity. It operates on an unlicensed, international frequency band. A radio that supports this standard can be accessed by means of the deviceís unique IEEE address and a channel. This implementation provides robust single-hop  communication between Sun SPOTs with clear channel checking, packet acknowledgement and retries.

The lowpan multiplexes traffic for up to 255 protocols over the radio connection between 
two Sun SPOTs or between a host application and a Sun SPOT. The intention of this design 
is that researchers can build their own protocol handlers above that level. However, to 
facilitate simple applications, two example protocols have been implemented using the 
GCF framework.

The generic connection framework (GCF) is part of J2ME, and defines a hierarchy of interfaces and classes that create connections (such as HTTP, datagram, or streams), and perform I/O. The GCF provides a generic approach to connectivity and is defined in the \texttt{javax.microedition.io} package. The GCF is based on standard uniform resource locators (URLs) to indicate the connection types to create (e.g., \texttt{http}, \texttt{file}, \texttt{sms}, 
\texttt{socket}, etc.).

The two example protocols provided are a streaming connection and a datagram-style 
connection. The URL for a streaming radio connection type is as follows,
where \texttt{address} is the unique IEEE (MAC) address of the Sun SPOT 
device to be communicating to, and \texttt{port} is the channel to be 
used: 

\begin{verbatim}
   radio://{address}:{port}
\end{verbatim}
%Channel 24 is a good channel to use to avoid interference
%between 802.11 and 802.15.4

An application can open a stream over the radio and 
then operate on that connection.  
For example, the following code will open a radio connection 
to Sun SPOT device 1020 on channel 42, and will output the
number 5 onto that radio stream: 

\begin{verbatim}
StreamConnection conn = (StreamConnection)
                Connector.open("radio://1020:42");
DataOutputStream output = 
                      conn.openDataOutputStream();
output.writeInt(5);
output.flush();
\end{verbatim}

The form of a datagram-style URL is as follows:

\begin{verbatim}
   radiogram://{address}|broadcast:{port}
\end{verbatim}

The radiogram protocol allows for broadcasting to multiple listeners in
addition to normal point-to-point communications.
The following code shows a radiogram being broadcast on channel 10:

\begin{verbatim}
DatagramConnection sendConn = (DatagramConnection) 
       Connector.open("radiogram://broadcast:10");
dg.writeUTF("My message");
sendConn.send(dg);
\end{verbatim}

and the following code receives a radiogram on channel 10: 

\begin{verbatim}
DatagramConnection recvConn = (DatagramConnection) 
                Connector.open("radiogram://:10");
recvConn.receive(dg);
String answer = dg.readUTF();
\end{verbatim}

An application can also send a radiogram to a specific 
Sun SPOT device. 
In the following example, the message ``Hello world'' is 
sent to the remote Sun SPOT at address 1020 on channel 42.
The connection established between both Sun SPOTs can then 
wait for receiving a datagram from the remote Sun SPOT:

\begin{verbatim}
StreamConnection conn = (StreamConnection)
             Connector.open("radiogram://1020:42");
Datagram dg = 
         conn.newDatagram(conn.getMaximumLength());
dg.writeUTF("Hello world");
conn.send(dg);      // send the datagram
conn.receive(dg);   // reuse the datagram to receive 
                    // from remote SPOT
\end{verbatim}

The Sun SPOT radio range is 90 meters. Other facilities exist to reduce 
transmission power so that Sun SPOTs only communicate with other Sun SPOTs 
in close range.


\subsection{Debugging Support}
\label{sec-debugging}

The Squawk JVM allows applications on SPOT devices to be debugged using Java
debugging environments that support the Java Debug Wire Protocol 
(JDWP)~\cite{JDWP}, such as NetBeans and JDB.
Due to strict memory constraints on the SPOT device, Squawk does
not implement JDWP fully on the device, but splits the work between three
components, as shown in Figure~\ref{fig-debug-arch}.
There is a {\em debug proxy} that runs on the developer's workstation, a
{\em debug agent} that is used to control the application
being debugged and communicate with the debug proxy, and a small
{\em debug agent support} in the VM itself.
The debug isolate and debug proxy communicate using a subset of the JDWP
known as the Squawk Debug Wire Protocol (SDWP).

\psfigbegin{Debugger_img.eps}{6cm}
\psfigend{fig-debug-arch}{The Squawk Debug Architecture}

This split architecture allows many of the memory-consuming components of a
Java Platform Debugger Architecture (JPDA)-compliant debugging environment to
be located on the development workstation instead of in Squawk, reducing
memory overhead.
In particular, the debug proxy on the workstation has access to the original
class files that went into the suites on the device, so it has access to line
number tables and method, field, and local variable names that may have been
stripped from the suites.

By default, Squawk is build with SDWP support, as well as a second 
low-level debugger to aid in debugging the hardware.  
Each of these debuggers adds a 10\% overhead on the interpreter 
loop, by checking whether a breakpoint has been set or not. 

%As shown in Table~\ref{table-debugdata}, keeping the debug information for 
%a suite Sun SPOT off device can cut the suite size by more than half. 
%In addition, the code to read this debug information and handle the relevant JDWP 
%commands is kept in the debug proxy, not on the Sun SPOT. Finally, by 
%packaging most of the debugger functionality as an isolate, we plan to
%explore the idea of deploying debugger functionality over the air, as 
%needed.
%
%\centertablebegin
%\begin{tabular}{|l|c|c|} \hline
% & On Device & On Workstation \\ \hline
%Squawk suite size & 264,496 & 677,640 \\
%Debug proxy jar size & - & 61,659 \\
%Debug isolate size & 66,268 & - \\ \hline
%\end{tabular}
%\centertableend{table-debugdata}{Suite File Size for Squawk VM With and Without the Debug Proxy.}


%\subsection{Integration with Desktop Applications}
%
%An Sun SPOT application may communicate with the Squawk desktop 
%VM, where returned information can be used to integrate with 
%a graphical interface, for example.  
%
%\doublecenterfigbegin
%\begin{verbatim}
%    // Listen for broadcasts from Spots        
%    DatagramConnection conn = (DatagramConnection) Connector.open ("radiogram://:52");
%    Datagram message = conn.newDatagram(conn.getMaximumLength());
%        
%    // Endlessly receive RSSI messages and drive the playNote application
%    while (true) {
%        try {
%            conn.receive(message);
%            String sender = message.getAddress();
%            String neighbour = message.readUTF();
%            int rssi = message.readInt();
%            playNote(sender,neighbour,64+rssi);
%            message.reset();
%        } catch (Exception ex) {
%            System.err.println("ERROR:" + ex.toString());
%        }
%    }
%\end{verbatim}
%\doublecenterfigend{fig-eg1}{Java Source Code to Listen for Broadcast 
%	Messages between Sun SPOTs on Channel 52 and Control a Sound Application 
%	Through Received RSSI Messages.} 
%
%Figure~\ref{fig-eg1} shows sample application code to listen for
%braodcasts from Sun SPOT devices and use the received signal 
%strength indicator (RSSI) for the last received operation in this
%connection to control a sound application (\texttt{playNote}) on 
%the speaker and graphically on the screen. 


\subsection{Authentication of Deployed Applications}
\label{sec-authentication}

In a split VM implementation, assurances need to be given 
as to the authenticity of the file to be run on-device, 
given that someone may tamper with the file to be deployed
from the translator (the desktop part of the VM) and such 
file may bring down the VM on the device (something not 
permitted in Java VMs). 

At suite creation time, a digital signature is applied to 
the suite using a private key (stored in the desktop).
The on-device VM checks the validity of the suite at 
deploy time, by authenticating the suite's signature using 
its public key (stored on-device).
If the signature is authenticated, the suite is installed
on the device. 
The public and private key pairs are generated at SDK 
installation time. 


\section{Experimental Results}
\label{sec-results}

We measured a variety of data on the Sun SPOT platform running medium-sized
Java applications that are used in the object oriented community;
Richards and Delta Blue; the Game of Life; a Math application that measures
integer and long computation performance; and the Java ME GrinderBench benchmarks 
used in cell phones; Chess, Crypto, KXML, Parallel, and PNG.

The Richards benchmark, a medium-sized language benchmark (400-500 lines)
simulates the task dispatcher in the kernel of an operating system.
The original program by Richards was written in BCPL.
Richards (gibbons) is Gibbons' translation into Java of a version of the benchmark 
in C; the code is not very object-oriented.
Richards (deutsch\_no\_acc) is Deutsch's object-oriented implementation of this
benchmark, where some of the task state is moved into separate objects.
There are several variations to each: 
Richards (gibbons\_final) defines classes and methods to be final where possible, 
Richards (gibbons\_no\_switch) replaces an 8-valued integer with three booleans 
and the switch in the scheduler is replaced by tests of these variables,
Richards (deutsch\_acc\_virtual) encapsulates all object state so that it is
accessed via methods,
Richards (deutsch\_acc\_final) virtual calls are made non-virtual by making 
classes and methods final where possible, and 
Richards (deutsch\_acc\_interface) has all classes inheriting from an interface 
class, simulating the most object-oriented framework-like behavior.  

DeltaBlue is a constraint solver benchmark of about 1000 lines of code.
The Game of Life simulates a cellular automaton which in turn simulates 
life of cells in a grid.
The Math benchmark tests the performance of integer and long operations. 

The benchmarks in the GrinderBench suite are as follows: 
Chess is a chess playing engine that is used to determine a set of chess moves, 
Crypto is a suite of algorithms including DES, DESede, IDEA, Blowfish, and Twofish,
measuring the performance of Java implementations in cryptographic transactions,
KXML measures XML parsing and/or DOM tree manipulation,
Parallel exercises a Java implementation's ability to perform its user interface
while interacting with the Internet, having multiple threads with some communication
threads running on the background, and 
PNG shows how fast a Java implementation can decode a PNG photo image of a 
typical size used on a mobile phone. 
The GrinderMark\texttrademark\ is a single number score that the Embedded 
Microprocessor Benchmark Consortium (EEMBC) provides, in addition to scores 
based on individual benchmark applications within the GrinderBench suite. 


\subsection{Static Footprint: Interpreted JVMs on the ARM}

We provide static footprint measurements for two different versions 
of Squawk: Squawk~1.0 (Squawk with CLDC~1.0 libraries, no debugging 
support, and no authentication of suite support), and Squawk~1.1 
(Squawk with CLDC~1.1 (floating point) libraries, SDWP debugging support, 
authentication of suite files, a partial implementation of the
Information Module Profile (IMP), and a second low-level debugger 
used to debug the hardware).  
Squawk~1.0 measurements were collected on an ARM7-based Sun SPOT 
with 256~KB of RAM and 2~MB of flash, 
and Squawk~1.1 measurements were collected on an ARM9-based Sun SPOT
with 512~KB of RAM and 4~MB of flash. 

The Squawk~1.0 interpreter executable is 80~KB and ran out of RAM. 
The rest of the VM and its libraries (CLDC~1.0, networking (IEEE 
802.15.4 media access control (MAC) layer), radio library to drive
the Chipcon radio, and hardware and sensor integration/control 
libraries) occupy 270~KB and ran out of flash. 
The demo sensor board library is 20~KB.

The Squawk~1.1 interpreter executable is 149~KB, 
the rest of the VM is 363~KB and the libraries (CLDC 1.1, networking
(IEEE 802.15.4 media access control (MAC) layer), radio library to drive the
Chipcon radio, and hardware and sensor integration/control libraries) occupy
156~KB.
The demo sensor board library is 20~KB.
%All components are installed into and executed from flash memory.
%The RAM footprint is 6~KB, the rest of the RAM is object memory.

\doublecentertablebegin
\begin{tabular}{|l|c|c|c|c|c|} \hline
VM              & Target  & Debugging Support & VM     & CLDC libraries & Sun SPOT libraries \\ \hline
Squawk~1.0      & ARM7tdmi & No                & 80~KB  & \multicolumn{2}{c|}{270~KB} \\
Squawk~1.1      & ARM920T  & Yes               & 149~KB & 363~KB & 156~KB \\
KVM~1.1         & ARMv4l   & No                & 131~KB & 504~KB & n/a \\ 
KVM\_d~1.1      & ARMv4l   & Yes               & 198~KB & 504~KB & n/a \\ \hline
\end{tabular}
\doublecentertableend{table-static-footprint}{Static Footprint of Interpreted JVMs Running on an ARM}

The size difference between Squawk~1.0 and Squawk~1.1 are due to the 
new functionality added to the VM in a short amount of time in order
to obtain Java ME 1.1 compliance.  
The codebase has not been cleaned up at this point in time. 

Table~\ref{table-static-footprint} shows comparisons of the Squawk 
footprint against the KVM CLDC~1.1 as compiled on an ARMv4l Linux
machine.  
%\remark{KVM classes.zip is 149KB and the unzipped version is 504KB adding all classfiles in the zip file}
Two versions of the KVM were compiled: KVM~1.1, the production build, 
excluding the ROMizing library, and KVM\_d~1.1, the same build including
KDWP support. 
%
% KVM compiled with no ROMizing: ROMIZING=0 (by default, KVM compiles with 
% ROMizing on).
% KVM no ROM   163KB
% KVM w/ ROM   264KB
% To add the KDWP support, ENABLE_JAVA_DEBUGGER=true and changes to the Makefile.
% KVM no ROM w/debugger  198KB
%
% Bill Pittori's numbers: 
%               ARM/linux   x86/linux
%
%KVM no ROM     131084      111652
%KVM w/ ROM     233764      213912
%KVM w/debugger 261980      238532
%

As seen in the table, Squawk~1.1's size is comparable to that of KVM~1.1 and 
KVM\_d~1.1.
Both Squawk~1.1 and KVM~1.1 provide CLDC~1.1 libraries, and both Squawk~1.1 and 
KVM\_d~1.1 provide debugging support that allows developers to debug Java 
programs with any JDWP-compatible debugger.


\subsection{Sun SPOT Memory Map}

Squawk runs out of flash memory on the (ARM9-based) Sun SPOT. 
The Sun SPOT flash is very low power with 1 million cycles/sector endurance.  
Out of the 4~MB of flash, one third is reserved for system code,
not all of which is in use, and two thirds are reserved for applications
and data.
Figure~\ref{fig-sunspot-flash} shows the distribution of memory for
the different components of the Squawk VM and associated libraries.
The system memory is configured as follows:
256~KB are reserved for the VM binary; 149~KB are in use at present,
512~KB are reserved for the VM suite; 363~KB are used, 64~KB of which
are for the debugger (the debug agent support in the VM),
448~KB are reserved for the library suite; 156~KB are in user,
and 64~KB are reserved and used by the bootloader.
The user memory has two application slots, each of 384~KB, and 
2,040~KB of data space available to applications. 

\psfigbegin{SunSPOT-flash.eps}{8cm}
\psfigend{fig-sunspot-flash}{Sun SPOT Flash Memory}

The Sun SPOT has 512~KB of SRAM. Less than 20\% of SRAM is
reserved for system memory, the rest is available for
application objects.
Figure~\ref{fig-sunspot-ram} shows the distribution of RAM memory.
The system memory is configured as follows:
16~KB are used by the page tables,
8~KB are used by the C stack,
8~KB are used by the GC stack,
16~KB are used by the C heap,
5~KB are used by C data, and
14~KB are used at startup.
The Java heap has 459~KB reserved for it.

\psfigbegin{SunSPOT-ram.eps}{8cm}
\psfigend{fig-sunspot-ram}{Sun SPOT RAM Memory}


\subsection{Performance: Interpreted JVMs on the ARM}

Table~\ref{table-performance-arm} shows results obtained from running Squawk~1.1 
on a 180~MHz 32-bit ARM920T Sun SPOT and the KVM~1.1 on a Sharp Zaurus 200~MHz 
32-bit ARMv4l Linux machine.
The Richards results for Squawk show that, as expected, performance degrades 
when virtual accessors and interfaces are introduced, and that the use of 
finals slightly improves the performance.  
The KVM shows a similar trend, though it performs much worse when virtual 
accessors and interfaces are introduced. 

\doublecentertablebegin
\begin{tabular}{|l|c|c|} \hline
Benchmark & Squawk (ARM920T 180~MHz) ms & KVM (ARMv4l 200~MHz) ms \\  \hline
Richards (gibbons) &                 1,296 & 980 \\
Richards (gibbons\_final) &          1,287 & 948 \\
Richards (gibbons\_no\_switch) &     1,412 & 1,262 \\
Richards (deutsch\_no\_acc) &        1,895 & 2,118 \\
Richards (deutsch\_acc\_virtual) &   3,314 & 6,002 \\
Richards (deutsch\_acc\_final) &     3,303 & 3,119 \\
Richards (deutsch\_acc\_interface) & 3,664 & 4,555 \\
DeltaBlue &                            792 & 470 \\
Game of Life &                       6,699 & 5,848 \\ 
Math int &                           6,764 & 4,077 \\ 
Math long &                         27,282 & 12,813\\ \hline 
\end{tabular}
\doublecentertableend{table-performance-arm}{Runtime Performance of Interpreted JVMs Running on an ARM}

Table~\ref{table-performance-grindermark} shows preliminary data for the 
performance of Squawk CLDC~1.1, compared to the KVM CLDC~1.0, 
when using the GrinderBench benchmarks.   
Squawk was run on a Sun SPOT ARM920T configured with 460~KB of heap.  
The KVM data was provided by Sun's Java ME team and was run on an 
ARM926EJ-S 60 MHz with 1~MB of Java heap.

\doublecentertablebegin
\begin{tabular}{|l|c|c|c|c|c|c|c|c|} \hline
VM & MHz & Heap & Chess & Crypto & KXML & Parallel & PNG & GrinderMark \\ \hline
Squawk 1.1 & 180 & 460~KB & 264 & 550 & 452 & 593 & 563 & 456.73 \\
KVM 1.0    &  60 &  1~MB  & 288 & 269 & 399 & 272 & 244 & 290.01 \\ \hline
\end{tabular}
\doublecentertableend{table-performance-grindermark}{GrinderBench Results} 

The data shows that the Squawk JVM performs in the general ballpark of KVM; i.e., 
that Squawk is comparable in performance to other interpreted JVMs despite the fact 
the JVM itself is mainly written in Java. 


\subsection{Suite File vs Class file Sizes}
\label{sec-results-filesizes}

We compare the sizes of Java .class files, Java compressed .class file (.jar
file), and the corresponding Squawk .suite file.  
The GrinderBench benchmarks require input data, such data is stored in 
resource files and can account for up to 20\% of the size of the combined 
.class and resource files.  
Table~\ref{table-sizes} shows the results of measuring the size of the
Java class files, compressed JAR file equivalent, and the corresponding Squawk 
suite files on a SPOT.

\doublecentertablebegin
\begin{tabular}{|l|r|r|r|r|r|r|} \hline
Benchmark & resources files & class files & JAR file & suite file & suite/(class+resource) & suite/jar \\ \hline
Richards (gibbons) &                 0 & 10,975 & 7,968  & 4,072 & 0.37 & 0.51 \\
Richards (gibbons\_final) &          0 & 10,981 & 7,973  & 4,080 & 0.37 & 0.51 \\
Richards (gibbons\_no\_switch) &     0 & 10,865 & 7,972  & 4,156 & 0.38 & 0.52 \\
Richards (deutsch\_no\_acc) &        0 & 16,560 & 11,637 & 6,044 & 0.36 & 0.52 \\
Richards (deutsch\_acc\_virtual) &   0 & 21,442 & 13,180 & 8,040 & 0.37 & 0.61 \\
Richards (deutsch\_acc\_final) &     0 & 21,440 & 13,191 & 8,040 & 0.37 & 0.61 \\
Richards (deutsch\_acc\_interface) & 0 & 22,632 & 14,131 & 8,040 & 0.39 & 0.63 \\
DeltaBlue &                          0 & 27,584 & 16,478 & 9,212 & 0.33 & 0.56 \\
Game of Life &                       0 & 8,467  &  5,444 & 3,472 & 0.41 & 0.64 \\
Math &                               0 & 2,224  &  2,122 & 1,264 & 0.57 & 0.60 \\ \hline
Subtotal &                          0 & 153,170 & 100,096 & 56,420 & 0.37 & 0.56 \\ \hline \hline
Chess &                        58,878 & 133,435 &  33,780 & 33,780 & 0.25 & 0.57 \\
Crypto &                        9,954 &  89,954 &  60,690 & 55,232 & 0.55 & 0.91 \\
KXML &                         19,109 & 111,346 &  66,318 & 57,732 & 0.44 & 0.87 \\
Parallel &                     38,731 &  99,747 &  49,848 & 49,848 & 0.50 & 1.29 \\
PNG &                          15,472 &  49,401 &  46,025 & 33,404 & 0.51 & 0.73 \\ \hline
Subtotal &                    142,144 & 483,883 & 256,661 & 229,996 & 0.37 & 0.90\\ \hline \hline
\end{tabular}
\doublecentertableend{table-sizes}{Class File, JAR, and Suite File Size Comparison in Bytes}

As seen in Table~\ref{table-sizes}, the compounded size of the suite files is 37\% 
the size of standard class and resources files, and 56\% and 90\% the size of 
compressed JAR files.  This latter figure is large when resource files are included
in the JAR file: JAR files compress both code and data, whereas the Squawk suite 
files only compress code by using a different representation of the Java 
bytecodes, and it does not compress data in any way. 
The first set of data is more representative of how Java applications get deployed
and used on a Sun SPOT. 


\subsection{Bytecode Optimization}

We present measurements of the bytecode optimizer as running the
Richards benchmarks on a desktop machine.  
Future work will integrate the bytecode optimizer into the Squawk 
for Sun SPOT release, without affecting debuggability of the optimized 
class files. 

\doublepsfigbegin{results-bytecode-optm.eps}{8cm}
\doublepsfigend{fig-bytecode-optimizations}{Results of Performing Bytecode Optimizations 
on the Richards Benchmarks}

Figure~\ref{fig-bytecode-optimizations} shows three pieces of data: 
applications running without any optimizations (the base line at 100\%), 
applications optimized with constant folding and constant and copy propagation,
and applications optimized with inlining and the previous optimizations. 
Note that the benefits of these optimizations will be better seen once
dead code elimination is implemented in Squawk. 

As seen in Figure~\ref{fig-bytecode-optimizations}, inlining of methods 
greatly improves the performance of the more object-oriented versions of
the Richards benchmark, reducing execution time to almost half when 
all object state is encapsulated and accessed via methods.  

%The bytecode optimizer was able to inline 15\% of method calls in the 
%bootstrap suite (counted statically), while decreasing the suite file 
%size by 2\%. Initial performance tests run on the desktop show that the 
%bytecode optimizer can increase performance up to 45\% in the case of the 
%more object-oriented version of the Richards benchmark. Other benchmarks 
%showed a 0-13\% performance improvement.


\subsection{Radio Performance}
\label{sec-radio-performance}

The radio communicates over-the-air at 250 kbps, and has an on-board
ring buffer large enough for at least one radio packet. If a second
packet is received before the first is read from the buffer, and both
packets are reasonably large, then the second packet will be lost.

The radio stack is designed so that data does not get copied through
the layers and consequently garbage is kept to a minimum. As a result,
applications that do little processing and generate minimal garbage
can deal with data at a continuous rate faster than 250 kbps, making
it possible to receive packets near continuously. This is especially
so for smaller packets where the buffer may hold more than one.

Also, for most practical applications, packets do not arrive
continuously and the presence of the buffer means the actual
processing rate is not as high. Nevertheless, for applications that do
significant processing and receive lots of packets in very quick
succession, packets may be lost.

If the packets being lost are point-to-point from another Sun SPOT,
that Sun SPOT will not get acknowledgements and will therefore
retry, making things slower.
If the pakcets are broadcast packets, then neither sender nor
receiver will know that the packets were missed.  This is the
nature of the IEEE 802.15.4 protocol.
If an application needs to guarantee delivery of broadcast
packets, an acknowledgement scheme at the application level is
needed.

It's worth pointing out that the same considerations apply if a
packet is lost due to radio interference.
Hence, a design to solve the interference problem will also
solve the overflow issues.


\section{Conclusions}
\label{sec-conclusions}

The Squawk Java VM is a small, mostly written in Java JVM 
that can easily be ported to run on other platforms.

Squawk was designed for small, resource constrained devices, 
and can run without need for an underlying operating system
on the Sun SPOT device.  
Squawk's architecture is that of a split VM architecture, 
where class loading is done on the desktop, and execution is
done on-device.  A file format known as suites is used to 
transfer applications from the desktop to the device. 
 
Facilities easily implemented in Squawk, such as the isolation
mechanism and isolate migration, are of much interest and use
in writing wireless sensor network applications, as isolates 
can be reified, and applications can be migrated from one 
device to another. 

Squawk provides a wireless API for the IEEE 802.15.4 protocol, 
which extends on the generic connection framework (GCF) and 
provides for radio and radiogram connection types.  The radiogram
connection allows for normal point-to-point communication, as 
well as broadcasting to multiple listeners.  

Results show that, even without performance tuning, the 
Squawk JVM performs reasonably well when compared to other 
interpreted JVMs, even though Squawk is mainly written in Java.  
Squawk's size is small despite implementing OS-level functionality 
to run on the bare metal, and the suite files it generates are 
about one third of the size of standard Java class files.


\acks

The original design and implementation of Squawk was
due to Nik Shaylor.

The initial drive for the Squawk on Sun SPOTs was due
to John Nolan.  John also implemented the demo sensor board 
library and some of the initial applications on the Sun SPOTs. 

We would like to thank Mario Wolczko, Greg Wright, Mikel Lujan, 
Mike Van Emmerik, and Randy Smith for comments and suggestions on 
ways of improving the presentation of this paper. 

Thanks also go to 
Bill Pittori for providing access to a Linux/X86 and ARMv4l machines
to compile and collect KVM data, 
Simon Long for collecting some of the data for this paper, 
Eric Arseneau and Martin Morissette for contributing to the CLDC~1.1 
conformance, 
Eric, Vipul Gupta and Christian Puhringer for the design and 
implementation of the suite signing architecture, and 
Nancy Snyder for the diagrams in this paper. 

For more information on Squawk refer to
\url{http://research.sun.com/projects/squawk}
and for more information on the Sun SPOT project refer to 
\url{http://www.sunspotworld.com}


%%
%% NOTE: have to include the bibliography inline in the paper
%%
%\bibliographystyle{alpha}
%\bibliography{vm}

\newcommand{\etalchar}[1]{$^{#1}$}
\begin{thebibliography}{GFWK02}

\bibitem[AAB{\etalchar{+}}99]{Alpe99}
B.~Alpern, D.~Attanasio, J.~Barton, A.~Cocchi, S.F. Hummel, D.~Lieber,
  M.~Mergen, T.~Ngo, J.~Shepherd, and S.~Smith.
\newblock Implementing {J}alape\~{n}o in {J}ava.
\newblock In {\em Proceedings {ACM SIGPLAN} Conference on Object-Oriented
  Programming Systems, Languages, and Applications (OOPSLA)}, Denver, Colorado,
  November 1999. {ACM} Press.

\bibitem[AAB{\etalchar{+}}00]{Alpe00}
B.~Alpern, C.R. Attanasio, J.J. Barton, M.G. Burke, P.~Cheng, J.D. Choi,
  A.~Cocchi, S.J. Fink, D.~Grove, M.~Hind, S.F. Hummel, D.~Lieber, V.~Litnivoc,
  M.F. Mergen, T.~Ngo, J.R. Russell, V.~Sarkar, M.J. Serrano, J.C. Shepherd,
  S.~Smith, V.C. Sreedhar, H.~Srinivasan, and J.~Whaley.
\newblock The {J}alape\~{n}o virtual machine.
\newblock {\em {IBM} System Journal}, 39(1), February 2000.

\bibitem[All01]{ELinux}
D.~Allison.
\newblock Embedded {L}inux applications: An overview.
\newblock
  \url{http://www-128.ibm.com/developerworks/linux/library/l-embl.html}, 2001.

\bibitem[CLD]{CLDC11}
{JSR} 139 - {CLDC} 1.1.
\newblock
  \url{http://jcp.org/aboutJava/communityprocess/final/jsr139/index.html}.

\bibitem[Cza00]{Czaj00}
G.~Czajkowski.
\newblock Application isolation in the {J}ava\texttrademark\ virtual machine.
\newblock In {\em Proceedings of the {ACM SIGPLAN} Conference on
  Object-Oriented Programming Systems, Languages and Applications {OOPSLA}},
  pages 354--366, Minneapolis, Minnesota, October 15--19 2000.

\bibitem[FHV03]{Flack03}
C.~Flack, T.~Hosking, and J.~Vitek.
\newblock Idioms in {OVM}.
\newblock Technical Report CSD-TR-03-017, Purdue University, Department of
  Computer Science, 2003.

\bibitem[GFWK02]{Golm02}
M.~Golm, M.~Felser, C.~Wawersich, and J.~Kleinoeder.
\newblock The {JX} operating system.
\newblock In {\em Proceedings of the {USENIX} Annual Technical Conference},
  pages 45--58, Monterey, CA, June 2002.

\bibitem[IKM{\etalchar{+}}97]{Inga97}
D.~Ingalls, T.~Kaehler, J.~Maloney, S.~Wallace, and A.~Kay.
\newblock Back to the future: The story of {S}queak, a practical {S}malltalk
  written in itself.
\newblock In {\em Proceedings {ACM SIGPLAN} Conference on Object-Oriented
  Programming Systems, Languages, and Applications (OOPSLA)}. {ACM} Press,
  October 1997.

\bibitem[JDW]{JDWP}
Java {P}latform {D}ebugger {A}rchitecture - {J}ava {D}ebug {W}ire {P}rotocol.
\newblock \url{http://java.sun.com/products/jpda/doc/jdwp-spec.html}.

\vfill\eject

\bibitem[JL96]{Jone96}
R.~Jones and R.~Lins.
\newblock {\em Garbage Collection--Algorithms for Automatic Dynamic Memory
  Management}.
\newblock John Wiley \& Sons, Chichester, England, 1996.

\bibitem[JSR05]{JSR121}
{JSR} 121 - {A}pplication isolation {API} specification.
\newblock
  \url{http://jcp.org/aboutJava/communityprocess/pfd/jsr121/index.html}, 2005.

\bibitem[Loh05]{Lohm05}
S.~Lohmeier.
\newblock Jini on the {J}node {J}ava {OS}.
\newblock Online article at \url{http://monochromata.de/jnodejni.html}, June
  2005.

\bibitem[Mus05]{Mustang}
Java 1.6 {M}ustang.
\newblock \url{https://mustang.dev.java.net/}, 2005.

\bibitem[PBF{\etalchar{+}}03]{Pala03}
K.~Palacz, J.~Baker, C.~Flack, C.~Grothorff, H.~Yamauchi, and J.~Vitek.
\newblock Engineering a customizable intermediate representation.
\newblock In {\em Proceedings of the Workshop on Interpreters, Virtual Machines
  and Emulators ({IVME})}, pages 67--76. {ACM} Press, June 2003.

\bibitem[SCS05]{Smit05}
R.~Smith, C.~Cifuentes, and D.~Simon.
\newblock Enabling {J}ava$^{TM}$ for small wireless devices with {S}quawk and
  {S}potworld.
\newblock OOPSLA Workshop Bringing Software to Pervasive Computing, Oct 16
  2005.

\bibitem[SMES01]{Schn01}
D.~Schneider, B.~Mathiske, M.~Ernst, and M.~Seidl.
\newblock Automatic persistent memory management for the {S}potless
  {J}ava\texttrademark\ virtual machine on the {P}alm connected organizer.
\newblock In {\em Proceedings of the {J}ava\texttrademark\ Virtual Machine
  Research and Technology Symposium ({JVM'01})}. {USENIX}, April 2001.

\bibitem[SSB03]{Shay03}
N.~Shaylor, D.~Simon, and B.~Bush.
\newblock A {J}ava virtual machine architecture for very small devices.
\newblock In {\em Proceedings of the Conference on Languages, Compilers, and
  Tools for Embedded Systems (LCTES)}, pages 34--41. {ACM} Press, June 2003.

\bibitem[TBS99]{Taiv99}
A.~Taivalsaari, B.~Bill, and D.~Simon.
\newblock The {S}potless system: Implementing a {J}ava\texttrademark\ system
  for the {P}alm connected organizer.
\newblock Technical Report {SMLI TR}-99-73, Sun Microsystems Research
  Laboratories, Mountain View, California, February 1999.

\bibitem[USA05]{Unga05}
D.~Ungar, A.~Spitz, and A.~Ausch.
\newblock Constructing a metacircular virtual machine in an exploratory
  programming environment.
\newblock In {\em Companion Proceedings to the {ACM SIGPLAN} Conference on
  Object-Oriented Programming Systems, Languages and Applications {OOPSLA}},
  pages 11--20. {ACM} Press, October 2005.

\end{thebibliography}

\end{document}
